\section{Introduction}

Internet of Things is a term with a wide range of interpretations \cite{Atzori20102787}, breafly, we can think of it as billions of devices, mainly resource constrained, which are interconnected between them and the Internet, in order to achieve a goal.

Many of this objectives require the use of a great amount of data, and thanks to organizations like WikiLeaks, people are aware of the implications of their data on the Internet, demanding more security and privacy for it. This includes not only the data shared with others, where one must trust they will keep it safe, but it is also the data collected about us and which we do not have direct control over it.

In traditional M2M (Machine to Machine) environments the issues about security and privacy have already been treated deeply, but in the IoT ecosystem, due to it's recent and fast growth, it lacks of those tools to autonomously solve these issues.

To address these problems of privacy in the Internet, a recent approach is the concept of \textit{strong anonymity}, that conceals our personal details while letting us continue to operate online as a clearly defined individuals \cite{stronganonymity}. To achieve it, we must address a way to perform authentication and authorization in the most privacy-friendly approach. Anonymous credentials and selective disclosure techniques allow us to control what information we reveal to others.

The most common type of ACS are ABCs (Attribute-Based Credentials), which can be thought of as a digital signature by an Issuer on a list of attribute-value pairs \cite{introCredIBM}.
The most straightforward way for a user to convince a Verifier of her list of attributes would be to simply transmit her credential to the Verifier.
With anonymous credentials, the user never transmits the credential itself, but rather uses it to convince the Verifier that her attributes satisfy certain properties - without leaking any information about the credential other than the chosen properties to display. 
Stronger even, apart from showing the exact value of an attribute, the user can even convince the
Verifier that some complex predicate over the attributes holds, e.g. that her birth date was more than 18 years ago, without revealing the real date.

With classical symmetric and asymmetric cryptography it seems rather impossible to create such credentials without an explosion of signatures over every possible combination of attributes. For this reason, current solutions rely on  Zero-Knowledge Proofs (ZKP), cryptographic methods that allow to proof knowledge of some data without disclosing it.

Based on ZKPs, IBM developed Identity Mixer\footnote{Identity Mixer - \url{https://www.zurich.ibm.com/identity_mixer/}}, %broken url
Idemix for short, a protocol suite for privacy-preserving authentication and transfer of certified attributes. It allows user authentication without divulging any personal data. Users have a personal certificate with multiple attributes, but they can choose how many to disclose, or only give a proof based on their values. Thus, no personal data is collected that needs to be protected, managed, and treated by third parties.

So far, Idemix or privacy-ABCs have been successfully applied to deal with traditional Internet scenarios, in which users can authenticate and prove their attributes against a service provider. However, due to the reduced computational capabilities of certain IoT devices, it has not been yet considered for IoT scenarios. As it is presented in the state of the art chapter, current implementations of Idemix are based on Java, which requires high computational and memory resources to be executed, and to the best of our knowledge, this is the first proposal that tries to apply an IoT solution for privacy-preserving authentication and authorization, based on Anonymous Credential Systems.

The design and implementation presented in this paper will allow constrained IoT devices to carry out all actions available in the Idemix protocol, being in control of the decisions to take in every step. This achieves oblivious interactions between any traditional machine using Idemix and the IoT devices, without having to adapt the protocol, or a subset it, to interact with these new entities.

\hfil

This document is structured as follows: Section \ref{ch:stateoftheart} provides a state of the art analysis through the history of Idemix and related works, analysing what is of the most interest for the IoT perspective. Section \ref{ch:design} presents a formal design of the proposed IoT and Idemix solution. Section \ref{ch:implementation} describes the PoC implementation developed. After any implementation, it is a must to validate it, as showed during the performance tests in section \ref{ch:validation}. Finally, our conclusions and lines for future work are described in section \ref{ch:conclusions}.
