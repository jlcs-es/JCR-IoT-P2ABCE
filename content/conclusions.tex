%************************************************
\section{Conclusions and Future Work}\label{ch:conclusions}
%************************************************

This paper has presented a generic privacy-preserving solution for the vast world of the Internet of Things. The IoT devices can operate as individual actors in the P2ABCE ecosystem, and when in need of performing computation offloading, the delegation server can also be a device considered into the IoT family.
The PoC implementation demonstrates that this project is actually feasible in constrained devices, not by performing a simulation of an IoT device, like in \cite{vanet} or \cite{alcaide2013anonymous}, but deploying it in a real IoT scenario.

As future work, we envisage to study the available solutions for delegation, e.g. using REST, CoAP, RPC, and for the security issues mentioned, in order to address the definition of the P2ABCE API to abstract the delegation process to other processes running in the IoT devices. 
We also expect to continue evolving the implementation, where it would be interesting to compare the execution of the current PoC to a version with cryptographic hardware acceleration, for example, using Atmel's chips for SHA, AES and secure memory.



\subsection*{Conflicts of Interest}

The authors declare that they have no conflicts of interest.

\subsection*{Acknowledgements}

The project leading to this application has received funding from IBM 2015 Faculty Award for Cyber Security and the European Union’s Horizon 2020 research and innovation programme under grant agreement No 700085 (ARIES project).
